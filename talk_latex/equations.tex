\documentclass{article}
\usepackage{mathtools}
\usepackage{physics}

\begin{document}

\makeatletter
\def\vec#1{\vb{#1}}
\def\floor#1{\lfloor #1 \rfloor}


\section{Lagrange}

\def\r{\vec{r}}
\def\dr{\vec{\dot{r}}}
\def\r@t{\r\qty(t)}
\def\dr@t{\dr\qty(t)}

\def\action{S}
\def\action@r{\action\qty[\r@t]}

\def\lagrange{\mathcal{L}}
\def\lagrange@t{\lagrange\qty(\r@t, \dr@t, t)}

\def\potential{V}
\def\potential@t{\potential\qty(\r@t, \dr@t, t)}

\def\int@ab{\int^{\vec{b}}_{\vec{a}} \dd{\r}}
\def\int@t{\int^T_0 \dd{t}}

\begin{align*}
    \action@r = 
    \int@ab \lagrange = 
    \int@t \lagrange@t =
    \int@t \frac m2 \norm{\dr@t}^2 - \potential@t
\end{align*}


\section{Splines}

\def\curve{\vec{C}}
\def\curve@lambda{\curve\qty(\lambda)}
\def\dcurve{\vec{\dot{C}}}
\def\dcurve@lambda{\dcurve\qty(\lambda)}

\def\segment{\curve^{(n)}}
\def\segment@lambda{\segment\qty(\lambda)}
\def\segment@nlambda{\segment\qty(\lambda - n)}
\def\segment@floor{\curve^{(\floor\lambda)}}
\def\segment@floor@lambda{\segment@floor\qty(\lambda - \floor\lambda)}
\def\dsegment{\dcurve^{(n)}}
\def\dsegment@lambda{\dsegment\qty(\lambda)}
\def\dsegment@nlambda{\dsegment\qty(\lambda - n)}

\def\basis_#1{f_#1}
\def\basis@lambda_#1{\basis_#1\qty(\lambda)}
\def\dbasis_#1{\dot{f}_#1}
\def\dbasis@lambda_#1{\dbasis_#1\qty(\lambda)}
\def\control{\vec{a}^{(n)}_j}

\def\cgrad{\grad^{(n)}_k}
\def\cdiv{\cgrad\vdot}

\begin{align*}
    \curve@lambda = 
    \segment@floor@lambda
\end{align*}
\begin{align*}
    \segment@lambda = 
    \sum_j \basis@lambda_j \control
\end{align*}
\begin{align*}
    \dsegment@lambda = 
    \sum_j \dbasis@lambda_j \control
\end{align*}
\begin{align*}
    \cdiv\curve@lambda = 
    \cdiv\segment@nlambda = 
    \basis@lambda_k
\end{align*}
\begin{align*}
    \cdiv\dcurve@lambda = 
    \cdiv\dsegment@nlambda = 
    \dbasis@lambda_k
\end{align*}


\section{Connection}

\def\curve@t{\curve\qty(\frac NT t)}
\def\dcurve@t{\dcurve\qty(\frac NT t)}

\def\r@alpha{\r\qty(T\alpha)}
\def\curve@alpha{\curve\qty(N\alpha)}
\def\dr@alpha{\dr\qty(T\alpha)}
\def\dcurve@alpha{\dcurve\qty(N\alpha)}

\def\int@lambda{\int^N_0 \dd{\lambda}}
\def\action@curve{\action\qty[\curve@lambda]}
\def\potential@lambda{\potential\qty(\curve@lambda, \frac NT \dcurve@lambda, \frac TN \lambda)}

\begin{align*}
    t \propto \lambda \Rightarrow t = \frac TN \lambda
\end{align*}
\begin{align*}
    \r@t = 
    \curve@t
\end{align*}
\begin{align*}
    \dr@t = 
    \frac NT \dcurve@t
\end{align*}
\begin{align*}
    \r@alpha = \curve@alpha \quad\forall \alpha \in \qty[0, 1] \xRightarrow{\alpha = 0}
    T = N \frac{\abs{\dcurve\qty(0)}}{\abs{\dr\qty(0)}}
\end{align*}
\begin{align*}
    \action@r = 
    \action@curve &= 
    %\frac TN \int@lambda \frac m2 \norm{\frac NT \dcurve@lambda}^2 + \potential@lambda \\ &= 
    \int@lambda \frac  m2 \frac NT \norm{\dcurve@lambda}^2 + \frac TN \potential@lambda
\end{align*}
\begin{align*}
    \cgrad\action@curve =
    \int@lambda m \frac NT \dcurve@lambda \qty(\cdiv\dcurve@lambda) +
    \frac TN \pdv{\potential}{\r} \qty(\cdiv\curve@lambda) +
    \pdv{\potential}{\dr} \qty(\cdiv\dcurve@lambda)
\end{align*}


\section{specifics}

\def\vbasis{\vec{f}}

\begin{align*}
    \vbasis = 
    \begin{pmatrix}
        1 - \lambda \\
        \lambda
    \end{pmatrix}
\end{align*}

\begin{align*}
    \vbasis = 
    \begin{pmatrix}
        \qty(1 - \lambda)^3 \\
        3 \qty(1 - \lambda)^2 \lambda \\
        3 \qty(1 - \lambda) \lambda^2 \\
        \lambda^3
    \end{pmatrix}
    \xrightarrow[\qq{Speicher-Reihenfolge}]{\qq{stetig differenzierbar}}
    \begin{pmatrix}
        \qty(1 - \lambda)^3 \\
        3 \qty(1 - \lambda)^2 \lambda \\
        \lambda^3 + 3 \qty(1 - \lambda) \lambda^2 \\
        6 \qty(1 - \lambda) \lambda^2
    \end{pmatrix}
\end{align*}


\end{document}